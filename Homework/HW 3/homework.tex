\documentclass{article}
\usepackage{amsmath} 
\usepackage{amsfonts}
\usepackage{amssymb}
\usepackage[parfill]{parskip}
\usepackage{esvect}
\usepackage[thinc]{esdiff}
\usepackage{mathtools}
\usepackage{physics}
\usepackage{amsthm}
\usepackage{xfrac}
\usepackage{enumitem}

\begin{document}

\title{MATH 430: HW 3}
\author{Jacob Lockard}
\date{10 February 2026}

\maketitle

\section*{Exercise 3.38}

\begin{quote}
    \itshape
    There is an isomorphism of $U_7$ with $\mathbb{Z}_7$ in which
    $\zeta = e^{i(2\pi/7)} \leftrightarrow 4$.
    Find the element in $\mathbb{Z}_7$ to which $\zeta^m$ must
    correspond for $m=0$,$2$,$3$,$4$,$5$, and $6$.
\end{quote}

We have:
\begin{align*}
    \zeta^0 = 1 \; &\leftrightarrow \; 0 \\
    \zeta^1 = \zeta \; &\leftrightarrow \; 4 \\
    \zeta^2 = \zeta \cdot \zeta \; &\leftrightarrow \; 4 +_7 4 = 1 \\
    \zeta^3 = \zeta^2 \cdot \zeta \; &\leftrightarrow \; 1 +_7 4 = 5 \\
    \zeta^4 = \zeta^3 \cdot \zeta \; &\leftrightarrow \; 5 +_7 4 = 2 \\
    \zeta^5 = \zeta^4 \cdot \zeta \; &\leftrightarrow \; 2 +_7 4 = 6 \\
    \zeta^6 = \zeta^5 \cdot \zeta \; &\leftrightarrow \; 6 +_7 4 = 3 \,.
\end{align*}

\section*{Exercise 3.42}

\begin{quote}
    \itshape
    \begin{enumerate}[label=\alph*.]
        \item Derive a formula for $\cos 3\theta$ in terms of $\sin \theta$
        and $\cos \theta$ using Euler's formula.
        \item Derive the formula $\cos 3\theta = 4\cos^3 \theta - 3 \cos \theta$
        from part (a) and the identity $\sin^2 \theta + \cos^2 \theta = 1$.
    \end{enumerate}
\end{quote}
\textbf{(a)} By Euler's formula,
\begin{align*}
    e^{i\cdot 3\theta} &= \bigl(e^{i\theta}\bigr)^3 \\
     \cos 3\theta + i\sin 3\theta &= (\cos \theta + i \sin \theta)^3 \\
    &= (\cos \theta + i \sin \theta)(\cos^2 \theta + i \cdot 2 \sin \theta \cos \theta - \sin^2 \theta) \,.
\end{align*}
We have:
\begin{align*}
    \cos 3\theta &= \cos^3 \theta - \sin^2 \theta \cos \theta - 2 \sin^2 \theta \cos \theta \\
    \cos 3\theta &= \cos^3 \theta - 3 \sin^2 \theta \cos \theta \,.
\end{align*}

\textbf{(b)} We have:
\begin{align*}
    \cos 3\theta &= \cos^3 \theta - 3 \sin^2 \theta \cos \theta \\
    &= \cos \theta(\cos^2 \theta - 3 \sin^2 \theta) \\
\end{align*}
???

\section*{Exercise 4.5}

We have:
\begin{align*}
    \sigma^{-1}\tau\sigma = \begin{pmatrix}
        1 & 2 & 3 & 4 & 5 & 6 \\
        2 & 6 & 1 & 5 & 4 & 3
    \end{pmatrix} \,.
\end{align*}

\section*{Exercise 4.8}

\begin{align*}
    \sigma^{100} = \sigma^{96+4} = \bigl(\sigma^6\bigr)^{16}\sigma^4
    = e^{16} \sigma^4 &= \sigma^4 \\
    &= \begin{pmatrix}
        1 & 2 & 3 & 4 & 5 & 6 \\
        6 & 5 & 2 & 1 & 3 & 4
    \end{pmatrix} \,.
\end{align*}

\section*{Exercise 4.10}

We have:
\begin{align*}
    \sigma &= (1, 3, 4, 5, 6, 2) \\
    \tau &= (1, 2, 4, 3)(5, 6) \\
    \mu &= (1, 5)(3, 4) \,.
\end{align*}

\section*{Exercise 4.13a}

We have:
\begin{align*}
    \mu\rho^2\mu\rho^8 = 
    \mu^2\rho^{12-2}\rho^8
    = \rho^{10+8}
    = \rho^{6} \,.
\end{align*}

\section*{Exercise 4.32}

Let $n \geq 3$, and denote $\Sigma = \{1, 2, \dots, n\}$. We will show that
the only element $\sigma$
of $S_n$ satisfying
$\sigma\gamma = \gamma\sigma$
for all $\gamma \in S_n$
is $\sigma = \iota$.
Let $\sigma \in S_n$
with $\sigma \neq \iota$. Then there's some $x \in \Sigma$ such that
$\sigma(x) = y \neq x$.
Let $\gamma : \Sigma \to \Sigma$
be such that $\gamma(y) =z$, $\gamma(z) = y$, and $\gamma(v) = v$
for all other $v \in \Sigma$. $\gamma$ is clearly bijective, so $\gamma \in S_n$.
Then,
\begin{align*}
    \sigma\gamma(x) = \sigma(x) = y \neq z = \gamma(y) = \gamma\sigma(x) \,.
\end{align*}
We've shown that for any $\sigma \neq \iota$ there's a $\gamma \in S_n$ such that
$\sigma\gamma \neq \gamma\sigma$.
We conclude that $\iota$ is the only element of $S_n$ satisfying
$\sigma\gamma = \gamma\sigma$ for all $\gamma \in S_n$.

\qedsymbol

\section*{Exercise 5.59}

Let $S$ by any subset of a group $G$.
We will show that
\begin{align*}
    H_S = \{s \in G \;\vert\; xs = sx \text{ for all } s \in S \}
\end{align*}
is a subgroup of $G$. Let $a, b \in H_S$, and let $s \in S$.
Then we have:
\begin{align*}
    (ab)s = a(bs) = a(sb) = (as)b = (sa)b = s(ab) \,.
\end{align*}
So $ab \in H_S$. Let $e$ be the identity in $G$, and let $s \in S$.
Then we have:
\begin{align*}
    es = s = se \,,
\end{align*}
by the definition of the identity. So $e \in H_S$. Since $H_S \subseteq G$,
we conclude that $H_S$ is a subgroup of $G$.

\qedsymbol

\end{document}