\documentclass{article}
\usepackage{amsmath} 
\usepackage{amsfonts}
\usepackage{amssymb}
\usepackage[parfill]{parskip}
\usepackage{esvect}
\usepackage[thinc]{esdiff}
\usepackage{mathtools}
\usepackage{physics}
\usepackage{amsthm}
\usepackage{xfrac}
\usepackage{enumitem}

\begin{document}

\title{MATH 430: HW 2}
\author{Jacob Lockard}
\date{28 January 2026}

\maketitle

\section*{Exercise 2.10 (Detailed)}

\begin{quote}
    \itshape
    Let $n$ be a positive integer and let $n\mathbb{Z} = \{nm \;\vert\; m \in \mathbb{Z}\}$.

    \begin{enumerate}[label=\alph*.]
        \item Show that $\langle n\mathbb{Z}, + \rangle$ is a group.
        \item Show that $\langle n\mathbb{Z}, + \rangle \simeq \langle\mathbb{Z}, + \rangle$.
    \end{enumerate}
\end{quote}

Let $n$ be a positive integer, let $pq$ denote ordinary integer multiplication
for any $p, q \in \mathbb{Z}$,
let $+$ be the ordinary addition operator on $\mathbb{Z}$, and
let $n\mathbb{Z} = \{nm \;\vert\; m \in \mathbb{Z}\}$.

\textbf{(a)} We will show that $\langle n\mathbb{Z}, * \rangle$ is a group,
where $* : n\mathbb{Z} \to n\mathbb{Z}$ is the function such that
$np*nq = p+q$ for all $np, nq \in n\mathbb{Z}$.



Let $np, nq \in n\mathbb{Z}$.
By distributivity in $\mathbb{Z}$,
\begin{align*}
    np * nq = np + nq = n(p + q) \,.
\end{align*}
By definition, $p, q \in \mathbb{Z}$, so $p + q \in \mathbb{Z}$, since $+$ is an operator
over the integers and thus has a codomain of $\mathbb{Z}$.
So, by definition, $np * nq = n(p+q) \in n\mathbb{Z}$. Since $np$ and $nq$
are arbitrary elements of the domain of $*$ and $np * nq$
is an element of its codomain, we conclude
that $*$ exists and is well-defined.

Let $np, nq, nr \in n\mathbb{Z}$.
By distributivity and additive associativity in $\mathbb{Z}$,
\begin{align*}
    (np * nq) * nr &= (np + nq) + nr \\
    &= \bigl(n(p + q)\bigr) + nr \\
    &= n\bigl((p + q) + r\bigr) \\
    &= n\bigl(p + (q + r)\bigr) \\
    &= np + \bigl( n(q + r)\bigr) \\
    &= np + ( nq + nr) \,.
\end{align*}
$np, nq, nr \in n\mathbb{Z}$ by assumption, so by definition,
\begin{align*}
    (np * nq) * nr = np + (nq + nr) = np * (nq * nr) \,.
\end{align*}
We conclude that the group associativity axiom holds for $\langle n\mathbb{Z}, * \rangle$.

Let $nm \in n\mathbb{Z}$, and let $0$ be the integer additive identity.
$0 \in n\mathbb{Z}$, since $0 = n(0)$.
$n \in \mathbb{Z}$ by assumption and $m \in \mathbb{Z}$ by definition,
so the closure of integer multiplication
ensures $nm \in \mathbb{Z}$. By the definition of $0$,
\begin{align*}
    nm * 0 = nm + 0 = nm = 0 + nm = 0 * nm \,.
\end{align*}
We conclude that the group identity axiom holds for $\langle n \mathbb{Z}, * \rangle$,
with $0 \in \mathbb{Z}$ being the identity.

Let $nm \in n\mathbb{Z}$. As shown above, $nm \in \mathbb{Z}$, so $nm$
has an integer additive inverse $-nm$.
The algebraic properties of the integers ensure that $-nm = (-1)nm = n(-1)m = n(-m)$.
Since $-m \in \mathbb{Z}$, by definition $-nm = n(-m) \in n\mathbb{Z}$.
By the definition of the integer additive inverse,
\begin{align*}
    nm * (-nm) = nm + (-nm) = 0 = (-nm) + nm = (-nm) * nm \,.
\end{align*}
We conclude that the group inverse axiom holds for $\langle n\mathbb{Z}, * \rangle$,
with the inverse of any $nm \in n\mathbb{Z}$ being its integer additive inverse $-nm$.
\qedsymbol

\textbf{(b)} We will show that $\langle n\mathbb{Z}, * \rangle \simeq \langle \mathbb{Z}, + \rangle$.

Let $f : n\mathbb{Z} \to \mathbb{Z}$ be defined by $f(nm) = m$.
Let $x \in n\mathbb{Z}$. By definition, we can write $x = np$ for some
$p \in \mathbb{Z}$. If we can also write $x = nq$ for some
$q \in \mathbb{Z}$, then since $n \neq 0$ the cancellation law
ensures $p = q$. So for any $x \in n\mathbb{Z}$, there's exactly one
way to write $x$ as a product of $n$ and an integer. Thus, $f$ exists.

For any $m \in \mathbb{Z}$, by definition $f(nm) = m$, so $f$ is surjective.
Let $np, nq \in \mathbb{Z}$. If $f(np) = f(nq)$, then by definition of the function
$p = q$ and thus $np = nq$, so $f$ is injective.

Let $np, nq \in \mathbb{Z}$. Then we have:
\begin{align*}
    f(np * nq) = f(np + nq) = f\bigl( n(p+q) \bigr) = p + q = f(np) + f(nq) \,,
\end{align*}
which follows from our definitions and from distributivity and closure in $\mathbb{Z}$.
\qedsymbol

\section*{Exercise 2.11}

Let $n \in \mathbb{N}$.
We will show that $\langle M, + \rangle$
is a group, where $M$ is the set of all real, diagonal $n \times n$ matrices and
$+$ is the ordinary matrix addition operator.

Let $A, B \in M$. For all $i, j \in \{ 1, 2, \dots, n \}$
with $i \neq j$, we have:
\begin{align*}
    (A + B)_{ij} = A_{ij} + B_{ij} = 0 + 0 = 0 \,,
\end{align*}
since $A$ and $B$ are diagonal. Since all the non-diagonal entries are zero,
$A + B$ is diagonal, and hence the codomain of $+$ is $M$.

Let $A, B, C \in M$. For all $i, j \in \{1, 2, \dots, n\}$,
we have:
\begin{align*}
    \bigl((A + B) + C\bigr)_{ij} &= (A+B)_{ij} + C_{ij} \\
    &= A_{ij} + B_{ij} + C_{ij} \\
    &= A_{ij} + (B + C)_{ij} \\
    &= \bigl(A + (B + C)\bigr)_{ij} \,,
\end{align*}
which follows from the definition of matrix addition and the associativity
of the reals.
Since corresponding entries of
$(A+B)+C$ and $A+(B+C)$ are equal, we conclude that the associativity axiom holds
for $\langle M, + \rangle$.

Let $\mathbf{0}$ be the matrix such that $M_{ij} = 0$
for all $i, j \in \{1, 2, \dots, n\}$.
$\mathbf{0} \in M$, since all entries, including the non-diagonal ones,
are zero.
Let $A \in M$.
Then we have:
\begin{align*}
    (A + \mathbf{0})_{ij} = A_{ij} + \mathbf{0}_{ijj} = A_{ij} + 0 = A_{ij} \,, \\
    (\mathbf{0} + A)_{ij} = \mathbf{0}_{ijj} + A_{ij} = 0 + A_{ij} = A_{ij} \,,
\end{align*}
by the definition of matrix addition and of $0 \in \mathbb{R}$.
So $A + \textbf{0} = A = \textbf{0} + A$, and $\textbf{0}$ satisfies
the identity axiom for $\langle M, + \rangle$.

Let $A \in M$. Let $-A$ be the matrix such that $A_{ij} = -A_{ij}$ for all
$i, j \in \{1, 2, \dots, n\}$. For all $i, j \in \{1, 2, \dots, n\}$ with
$i \neq j$, we have:
\begin{align*}
    (-A)_{ij} = -(A)_{ij} = -0 = 0 \,,
\end{align*}
since $A$ is diagonal. So $-A \in M$. We have:
\begin{align*}
    \bigl(A + (-A)\bigr)_{ij} = A_{ij} + (-A)_{ij} = A_{ij} + (-A_{ij}) = 0 \,, \\
    \bigl((-A) + A\bigr)_{ij} = (-A)_{ij} + A_{ij} = (-A_{ij}) + A_{ij} = 0 \,,
\end{align*}
by the definition of matrix addition and of additive inverses in $\mathbb{R}$.
So $A + (-A) = \textbf{0} = (-A) + A$, and we've shown that the inverse
axiom holds for $\langle M, + \rangle$.
\qedsymbol

\section*{Exercise 2.17}

Let $n \in \mathbb{N}$. We will show that $\langle M, \times \rangle$ is a group,
where $M$ is the set of all real $n \times n$ upper-triangular matrices with determinant $1$,
and $\times$ is the ordinary matrix multiplication operator.
We also denote $A \times B$ like $AB$ for any real $n \times n$ matrices $A$ and $B$.

Let $A, B \in \mathbb{M}$. Let $i, j \in \{1, 2, \dots, n\}$ with $i > j$. By the definition of matrix multiplication,
\begin{align*}
    (AB)_{ij} = \sum_{k=1}^n A_{ik} B_{kj}
    = \sum_{k=1}^{i-1} A_{ik}B_{kj} + \sum_{k=i}^n A_{ik}B_{kj} \,.
\end{align*}
If $k \in \{1, 2, \dots, i-1\}$, then $i > k$ and $A_{ik} = 0$ since
$A$ is upper triangular. So the first sum is zero:
\begin{align*}
    \sum_{k=1}^{i-1} A_{ik} B_{kj} = \sum_{k=1}^{i-1} 0(B_{kj}) = 0 \,.
\end{align*}
If $k \in \{i, i+1, \dots, n\}$, then $k \geq i > j$ and $B_{kj} = 0$ since
$B$ is upper triangular. So the second sum is zero:
\begin{align*}
    \sum_{k=i}^{n} A_{ik} B_{kj} = \sum_{k=i}^n B_{kj}(0) = 0 \,.
\end{align*}
We've shown that for any $i, j \{1, 2, \dots, n\}$ with $i > j$, we have
$(AB)_{ij} = 0$. So $AB$ is upper triangular.
Further, we have:
\begin{align*}
    \det(AB) = \det(A) \det(B) = 1 \cdot 1 = 1 \,,
\end{align*}
so the determinant of $AB$ is $1$. We conclude that the codomain of
$\times$ is $M$.

Any introductory linear algebra text will demonstrate that $\times$ is
associative over $M$, and that there exists an identity matrix $I$
for the set of $n \times n$ matrices. This identity is upper triangular
and has determinant $1$, so we are assured it is in $M$.

Finally, let $A \in M$. Since $\det A = 1$, we know it is invertible. Denote
its inverse $A^{-1}$. We know $\det A^{-1} = (\det A)^{-1} = 1$, and for the
purposes of this proof, we'll assume
that $A^{-1}$ is upper triangular.
\qedsymbol

\section*{Exercise 2.28}

\begin{quote}
    \itshape
    An element $a \neq e$ in a group is said to have order $2$ if $a * a = e$.
    Prove that if $G$ is a group and $a \in G$ has order $2$, then for any
    $b \in G$, $b'*a*b$ also has order $2$.
\end{quote}

Let $G$ be a group. Let $a \in G$ have order $2$, and let $b \in G$. We have:
\begin{align*}
    (b'*a*b)*(b'*a*b) &= b'*a*(b*b')*a*b && \quad\text{associativity} \\
    &= b'*a*e*a*b && \quad\text{inverses} \\
    &= b'*a*a*b && \quad\text{associativity, identity} \\
    &= b'*e*b && \quad\text{assoc.; since $a$ is order $2$} \\
    &= b'*b && \quad\text{associativity, identity} \\
    &= e \,. && \quad\text{identity} \\
\end{align*}
\qedsymbol

\section*{Exercise 2.31}

\begin{quote}
    \itshape
    Prove that a group has exactly one idempotent element.
\end{quote}

Let $\langle G, * \rangle$ be a group. Let $a$ be an idempotent
for $*$ in $G$. Then,
\begin{align*}
    a*a &= a \\
    a*a*a' &= a*a' \\
    a*e &= a*a' \\
    a &= e \,.
\end{align*}
Since the group identity is unique, the idempotent must also be unique.
\qedsymbol

\section*{Exercise 2.32}

\begin{quote}
    \itshape
    Show that every group $G$ with identity $e$ and such that $x*x=e$ for all $x\in G$ is abelian.
\end{quote}

Let $a, b \in G$. Then,
\begin{align*}
    (a*b)*(a*b) &= e \\
    a*b*a*b &= e \\
    a*a*b*a*b &= a*e \\
    b*a*b &= a*e \\
    b*b*a*b &= b*a*e \\
    a*b &= b*a \,.
\end{align*}
\qedsymbol

\section*{Exercise 2.33}

\begin{quote}
    \itshape
    Let $G$ be an abelian group and let $c^n = c*c*\dots*c$ for $n$ factors
    $c$, where $c \in G$ and $n \in \mathbb{Z}^+$. Give a mathematical induction proof
    that $(a*b)^n = (a^n)*(b^n)$ for all $a, b \in G$.
\end{quote}

Let $\langle G, * \rangle$ be an abelian group.
For every $n \in \mathbb{Z}^+$, let $P_n$ be the statement
that $(a*b)^n = (a^n)*(b^n)$ for all $a, b \in G$.
We will show that $P_n$ holds for all $n \in \mathbb{Z}^+$.

$P_1$ holds, since for all $a, b \in G$,
\begin{align*}
    (a*b)^1 = a*b = (a^1)*(b^1) \,.
\end{align*}

Now assume that $P_n$ holds for some $n \in \mathbb{Z}^+$. We have, for all $a, b \in G$:
\begin{align*}
    (a*b)^{n+1} &= (a*b)^n * (a*b) && \quad\text{definition} \\
    &= (a^n)*(b^n)*a*b && \quad\text{assumption} \\
    &= (a^n)*a*(b^n)*b && \quad\text{abelian} \\
    &= (a^{n+1})*(b^{n+1}) \,. && \quad\text{definition}
\end{align*}
We've shown that $P_1$ holds and that $P_n$ implies $P_{n+1}$ for all $n \in \mathbb{Z}^+$.
By induction, we conclude that $P_n$ holds for all $n \in \mathbb{Z}^+$.
\qedsymbol

\section*{Exercise 2.36}

\begin{quote}
    \itshape
    Let $G$ be a group with a finite number of elements. Show that for any
    $a \in G$, there exists an $n \in \mathbb{Z}^+$ such that $a^n = e$.
\end{quote}

Let $a \in G$. Let $f : \mathbb{N} \to G$ be defined like $f(n) = a^n$.
$\mathbb{N}$ is infinite and $G$ is finite, so $|\mathbb{N}| > |G|$, which means
$f$ is not injective.
So there exist $n, m \in \mathbb{N}$ such that $a^n = f(n) = f(m) = a^m$ and
$n \neq m$. Assume without loss of generality that $m > n$. We have:
\begin{align*}
    a^n &= a^m \\
    a^n &= a^{(m-n)+n} \\
    a^n &= a^{m-n}*a^{n} \\
    a^n * (a^{-1})^n &= a^{m-n} * a^n * (a^{-1})^n \\
    (a * a^{-1})^n &= a^{m-n} * (a * a^{-1})^n \\
    (e)^n &= a^{m-n} * (e)^n \\
    e &= a^{m-n} * e \\
    e &= a^{m-n} \,.
\end{align*}
But $m - n > 0$, so $m - n \in \mathbb{Z}^+$. We've thus shown that
$m - n$ has the desired properties.

\qedsymbol

\section*{Exercise 2.38}

\begin{quote}
    \itshape
    Let $G$ be a group and let $a, b \in G$. Show that $(a*b)' = a'*b'$
    if and only if $a*b=b*a$.
\end{quote}
If $(a*b)' = a'*b'$,
\begin{align*}
    (a*b)' * a*b &= e \\
    a'*b' * a*b &= e \\
    a*b &= b*a \,.
\end{align*}
If $a*b = b*a$,
\begin{align*}
    (a*b)' * a*b &= e \\
    (a*b)' * b*a &= e \\
    (a*b)' &= a' * b' \,.
\end{align*}

\qedsymbol

\end{document}